
\begin{center}
    {\Large \textbf{Aprendizado de máquina e inteligência artificial em física} \hspace{0.5cm}}\\
    \vspace{0.3 cm}
	\vspace{0.3 cm}
	\text{Aluno\_name N°USP: }\\
        \text{Giulya Souza dos Santos N°USP: 14740109}\\
        \text{Pierre Augusto Ré Martho N°USP: 11298622}\\
	\vspace{0.3 cm}
	\textbf{2025}
\end{center}    
\rule{\textwidth}{0.5pt}

\section{Contexto}\label{sec:contexto}

A determinação do \textit{redshift} de galáxias é uma etapa fundamental das pesquisas em astronomia extragaláctica e cosmologia observacional. Trata-se de um parâmetro essencial para o estudo da estrutura em larga escala do universo, o mapeamento tridimensional da distribuição da matéria e investigação da história evolutiva das galáxias [REF]. O \textit{redshift} ($z$) pode ser entendido de forma simplificada como o deslocamento para o vermelho das linhas espectrais devido à expansão do universo. 

A técnica mais precisa para determinar o \textit{redshift} de um objeto é a espectroscopia, pois permite a identificação de características espectrais específicas, como linhas de absorção e emissão, possibilitando medidas confiáveis do deslocamento para o vermelho. Logo, tradicionalmente, os \textit{redshifts} espectroscópicos são considerados valores de referência devido à alta precisão do método [REF]. No entanto, esse procedimento depende de tempos de observação longos, instrumentação sofxisticada e é limitado pela sensibilidade de telescópios [REF], tornando inviável sua aplicação em larga escala.

Levantamentos fotométricos consistem em observações que fornecem imagens dos objetos em diferentes bandas, sem o detalhamento espectral no fluxo de luz obtido. A fotometria permite observar simultaneamente múltiplos objetos em diferentes bandas, resultando num volume de dados significativamente grande [REF]. Por este motivo, torna-se necessário o desenvolvimento de técnicas para estimar \textit{redshifts} a partir de dados fotométricos, que embora apresentem menor resolução, são muito mais abundantes e acessíveis que dados espectroscópicos.

Existem dois métodos principais para a determinação de \textit{redshifts} fotométricos (foto-$z$'s) descritos na literatura. O primeiro é o método baseado na Distribuição Espectral de Energia (SED), que compara a SED dos objetos observados com espectros modelo de uma variedade de objetos em diferentes \textit{redshifts} [REF]. O segundo é o método empírico, que se baseia no aprendizado de máquina e utiliza amostras de treinamento contendo \textit{redshifts} espectroscópicos [REF], sendo neste contexto que se insere este projeto.
